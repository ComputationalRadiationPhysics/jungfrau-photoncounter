\documentclass[a4paper]{article}
\usepackage{authblk}
\usepackage[backend=bibtex]{biblatex}
\usepackage{hyperref}
\usepackage{txfonts}
\usepackage{titling}
\usepackage{graphicx}
\usepackage{subcaption}
\usepackage[a4paper, margin={2.5cm, 1.7cm}]{geometry}
\usepackage{mwe}
\usepackage{filecontents}
\usepackage{changes}
\usepackage{lipsum}% <- For dummy text
\definechangesauthor[name={Heide}, color=blue]{Heide}
\setremarkmarkup{(#2)}

\bibliography{database.bib}

\begin{document}
\title{Fast processing of Jungfrau detector data}
\author{Jonas Schenke$^{1,2}$, 
Florian Warg$^{1,2}$, 
Anna Bergamaschi$^3$,
Martin Br\"uckner$^3$,\\
Michael Bussmann$^1$,
Carlos Lopez-Cuenza$^3$,
Aldo Mozzanica$^3$,\\
Sophie Redford$^3$,
Bernd Schmitt$^3$,
Heide Mei{\ss}ner}

\affil[1]{HZDR}
\affil[2]{TU Dresden}
\affil[3]{PSI}



\date{}

\renewcommand\Affilfont{\itshape}



\maketitle
{\bf Abstract}\\
...text...\\

{\bf Keywords:} Photon pixel detector, fast data processing, GPU programming, Alpaka...



\section{Introduction}

Increasing data rates during FEL experiments require dedicated detectors as well as advanced methods for fast data processing\\

Jungfrau detector (\cite{jungfrau1}, \cite{jungfrau2}, \cite{jungfrau3}, \cite{Mozzanica_2016}): pixel detector with gain switching scheme for large range of photon rates (single pixel to photon bunches)\\

charge integrating, meaning that detector output must be corrected before the number of photons per pixel per frame is retrieved.\\

online data conversion: calculate energy and number of photons from detector response for each pixel using continuously updated pedestal maps and gain maps calculated in dedicated lab-based calibration procedure~\cite{jungfrau2}\\

Find clusters\\

Hardware-independent computation\\

Different numbers of modules, parallel processing\\
	
GPU / Alpaka: \cite{Matthes17}: portable, parallel and scalable code\\

Related work\\

...\\

In the following, we describe ....\\


\section{Methods}
\subsection{Abilities and applications of the Jungfrau Detector (PSI)}

...




\begin{figure}[h!]
\centering
\includegraphics[width=0.30\textwidth]{jungfraudetector.jpg}
\caption{Jungfrau detector}
\label{fig:jfdetector}
\end{figure}


\subsection{Data processing algorithm (PSI)}
conversion from detector data to energy and number of photons including pixel mask, pedestal calculation, pedestal tracking, pedestal correction, gain correction, conversion to number of photons (in the case of monochromatic incident radiation)\\

 summation of frames\\

 clustering (reference?)\\

\subsection{Alpaka implementation of fast data processing (Jonas, Florian)}


\subsection{Benchmark tests (Jonas, Florian)}
Design, objectives, and evaluation of tests

\section{Results}
\subsection{Achieved improvements (Jonas)}
Results of tests of software parts on various computing hardware using suitable Alpaka backends\\

Where are the bottlenecks\\

Available capacity on GPUs\\

Best system

\subsection{Experiences from practical application of improved code (PSI)}
Application results

\section{Conclusions and Outlook}

Is presented method applicable / useful for other detectors, e.g. AGIPD? \\

calculation on FPGAs in the future


\section{Acknowledges}
Thanks to Alpaka developers.....\\

This project has received funding from the European Union's Horizon 2020 research and innovation programme under grant agreement No 654220 (EUCALL).

\newpage

\begin{sloppypar}
\printbibliography
\end{sloppypar}
\end{document}
